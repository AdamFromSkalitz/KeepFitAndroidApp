\documentclass[12pt]{report}

%\usepackage[utf8]{inputenc}
%\usepackage[T1]{fontenc}
%\usepackage{geometry}
%\geometry{a4paper}
\usepackage[english]{babel}

\usepackage{listings}
\usepackage{color}
\usepackage[usenames,dvipsnames]{xcolor}
\usepackage{graphicx}
\usepackage{placeins} 
\usepackage{flafter}
\usepackage[hidelinks]{hyperref}

\usepackage{tikz}
\usetikzlibrary{shapes.geometric, arrows}
\tikzstyle{start} = [rectangle, rounded corners, minimum width=3cm, minimum height=1cm,text centered, draw=black, fill=red!30]
\tikzstyle{action} = [trapezium, trapezium left angle=70, trapezium right angle=110, minimum width=3cm, minimum height=1cm,text width=3cm, text centered, draw=black, fill=blue!30]
\tikzstyle{thought} = [rectangle, minimum width=3cm, minimum height=1cm, text centered, text width=3cm, draw=black, fill=orange!30]
%\tikzstyle{decision} = [diamond, minimum width=3cm, minimum height=1cm, text centered, text width=3cm, draw=black, fill=green!30]
\tikzstyle{decision} = [ellipse,minimum width=1cm, minimum height=1cm, text centered, text width=3cm, draw=black, fill=green!30]
\tikzstyle{arrow} = [thick,->,>=stealth]

\definecolor{mygray}{rgb}{0.5,0.5,0.5}
\definecolor{mymauve}{rgb}{0.58,0,0.82}
%
%
%\\ start a new paragraph.
%\\* start a new line but not a new paragraph.
%\- OK to hyphenate a word here.
%\cleardoublepage flush all material and start a new page, start new odd numbered page.
%\clearpage plush all material and start a new page.
%\hyphenation enter a sequence pf exceptional hyphenations.
%\linebreak allow to break the line here.
%\newline request a new line.
%\newpage request a new page.
%\nolinebreak no line break should happen here.
%\nopagebreak no page break should happen here.
%\pagebreak encourage page break.

%\lstset{ %
%  backgroundcolor=\color{white},   % choose the background color; you must add \usepackage{color} or \usepackage{xcolor}
%  basicstyle=\footnotesize,        % the size of the fonts that are used for the code
%  breakatwhitespace=false,         % sets if automatic breaks should only happen at whitespace
%  breaklines=true,                 % sets automatic line breaking
%  captionpos=b,                    % sets the caption-position to bottom
%  commentstyle=\color{mygreen},    % comment style
%  deletekeywords={...},            % if you want to delete keywords from the given language
%  escapeinside={\%*}{*)},          % if you want to add LaTeX within your code
%  extendedchars=true,              % lets you use non-ASCII characters; for 8-bits encodings only, does not work with UTF-8
%  frame=false,                    % adds a frame around the code
%  keepspaces=true,                 % keeps spaces in text, useful for keeping indentation of code (possibly needs columns=flexible)
%  keywordstyle=\color{blue},       % keyword style
%  language=C,                 % the language of the code
%  morekeywords={*,...},            % if you want to add more keywords to the set
%  numbers=left,                    % where to put the line-numbers; possible values are (none, left, right)
%  numbersep=7pt,                   % how far the line-numbers are from the code
%  numberstyle=\tiny\color{mygray}, % the style that is used for the line-numbers
%  rulecolor=\color{black},         % if not set, the frame-color may be changed on line-breaks within not-black text (e.g. comments (green here))
%  showspaces=false,                % show spaces everywhere adding particular underscores; it overrides 'showstringspaces'
%  showstringspaces=false,          % underline spaces within strings only
%  showtabs=false,                  % show tabs within strings adding particular underscores
%  stepnumber=2,                    % the step between two line-numbers. If it's 1, each line will be numbered
%  stringstyle=\color{mymauve},     % string literal style
%  tabsize=2,                       % sets default tabsize to 2 spaces
%  title=\lstname                   % show the filename of files included with \lstinputlisting; also try caption instead of title
%}

\lstdefinestyle{customc}{
  numbers=left,
  numberstyle=\scriptsize\color{mygray},
  %belowcaptionskip=1\baselineskip,
  breaklines=true,
  frame=false,
  xleftmargin=\parindent,
  language=C,
  showstringspaces=false,
  basicstyle=\footnotesize\ttfamily,
  keywordstyle=\bfseries\color{Fuchsia},
  commentstyle=\itshape\color{green!40!black},
  identifierstyle=\color{blue},
  stringstyle=\color{mygray},
  title=\lstname  
}

\lstset{escapechar=@,style=customc,label=DescriptiveLabel}



%\paperheight=900pt
\voffset=-80pt
\textheight=650pt
\hoffset = -20pt
%\marginparwidth=100pt
%\textwidth =50cm
%\usepackage{a4wide}

%\usepackage[top=3cm, bottom=3cm, left=3cm, right=60cm]{geometry}

\usepackage{titlesec}
\titleformat{\chapter}{}{}{0em}{\bf\LARGE}
%\titleformat{\section}{}{}{0em}{\bf\Large}
\titleformat{\subsubsection}{}{}{0em}{\bf\large\bfseries}

\usepackage{layout}
\renewcommand*{\familydefault}{\ttdefault}
\renewcommand{\lstlistingname}{Code Example}
%\newcommand*{\myfont}{\rmdefault{ppl}\selectfont}

\title{KeepFit Android Application}
\author {Adam Kidd\\
	University of Strathclyde\\
	5th Year\\
	CS551}
\date{20/03/2017}

%comments

\begin{document}
\maketitle



% Provide overview on the state of the art(with references) in the relevant areas of technology 
% Discuss how these technical considerations have bearing on the topic
% Identify key issues and present the authors position on these issues with justification based on the background survey and any other factors

% Relevant Background survey - 40%
% Account of main topical issues - 20%
% Discussion - 20%
% Conclusions - 10%
%Organisation and presentation - 10%

\chapter{}


\subsubsection{Stages attempted}

All of the possible stages were attempted and are detailed below.

\subsubsection{Components}

Bottom Navigation

The app makes use of the bottom navigation bar for the user's navigation of the main screen set. There is the recommend number of top-level destinations with icons that all fit on the bar with equal spacing. 
\newline
\newline
Raised Buttons

They are rectangular-shaped buttons which use the app’s colour scheme and have a shadow. The Statistics and History screens have many elements and is a busy screen so raised buttons, in place of flat buttons, were used to give more prominence to the actions represented by the buttons. 
\newline
\newline
Dropdown Button

The arrow in a Dropdown and items in the list ripple when pressed and the list, when opened, starts from the top and expands downward as specified by Google.
\newline
\newline
Flat Buttons 

They are essentially just text but made more prominent with a different text size or colour than regular text. Flat buttons are printed on material and do not cast a shadow.
 \newline
 \newline
 Floating action button
 
It is circular material button that floats above the material, casts a shadow and displays an ink reaction or ripple on press. It is used for the primary actions of the app which should be positive like create and not negative like delete. 
 \newline
 \newline
Dialogues

The app uses full-screen dialogues for the addition of goals and steps, and the edit of goals. These actions require the full attention of the user and are a way of providing additional layers of material without significantly increasing the app’s perceived z-depth or visual noise.
 \newline
 \newline
Options Menu

The menu appears in the top right of app. It contains Settings, Statistics and Delete history. The label of a tile concisely and accurately reflects the items within the menu. The menu has only static content so the order of the list is from most to least frequently used in a downward order.
 \newline
 \newline
Snackbars

The snackbar appear when a user deletes a goal from the list and they are given the option to undo the delete. It contains a single line of text, no icons. It animates upwards from the bottom edge of the screen and does not cover the bottom navigation bar.
 \newline
 \newline
Toasts

They are primarily used for keeping the user notified about the state of the app. They display at the bottom of the screen for a short time after an action has occurred. 

\subsubsection{Patterns}

Confirm and Acknowledge

Components such as Toasts and Snackbars and visual feedback such as an item disappearing from a list are used throughout the app to keep the user informed and highlight they are in control.
Alert dialogue to confirm they want to delete history as this option is irreversible. 
Snackbars appears to tell user a goal has been deleted but it can be reversed if desired.
Toasts appear to tell the user something has changed app's backend state.
 \newline
 \newline
Date format 

This format of dates are the date in number format and the month as a written word, 19 march, to reduce user confusion as there are multiple date formats but this one translates to both.
 \newline
 \newline
Empty States

When there are simply no goals to display, in the history, goals or statistics screens, or no active goal for the home screen. The app displays a special screen to show that this lack of goals is normal and not an error. It also explains the situation, with a positive tone, so the user knows how fix it. 
 \newline
 \newline
Errors

When a error occurs, the user is warned with a Toast i.e. when goals/steps weren't recorded properly.
In the add goal and step dialogues if the user hasn't completed the necessary fields, the text view shows are error and explanation why it has appeared.
 \newline
 \newline
Notifications

The app has notifications that appear in the phone's status bar/notification drawer when the goal value is reached. The notification contains app name, dynamic content and title. If the notification is pressed the app opens. The user has the option to turn off notifications.


\section{Implemented Functionality}

The app was built around a main activity with the Home, History and Goals fragments being swapped in and out. The tool bar contains the name of the current screen and a menu that allows the user to go to the Settings, Statistics and Delete History activities. All of the goal data is stored in various tables in the SQLite database. The dialogues for adding goal, steps and editing goals are actually activities but the theme is changed to a Dialog one. There is an alarm service set for 23:59 and a corresponding broadcast receiver which will add the floating progress into the active goal then copy it from the goal list into the history and reset the floating progress. The alarm is set when the user opens the app for the first time and there is a DeviceBootReciver class so that the alarm is set if the device reboots but the user hasn't opened the app. The floating progress is a table in the datebase that holds the progress so it is not associated with any goal until the day has ended. The Grid View, statistics blocks, and Recycler View, goals list, have their own custom adapters which define the behaviour and appearance of the contents of the lists.


\subsection{Basic activity goal setting and recording}

To add a goal, the user presses the FAB, made using Clans FAB, which expands into two options, one is add goals. Once pressed the dialogue appears, figure \ref{fig:addgoal}. This then saves the goal into the sqlite database. In the goals fragment, figure \ref{fig:goalviewfab}, the list of goals is retrieved from the DB and displayed to the user in a scrollable list, all the information about each goal is displayed in the cards. There are three buttons to the right of each entry and they can be used to delete, edit or make active or nonactive a goal. The icons are ordered so delete is in the middle of the entry so the user has to make more effort to press it as it is has a larger side effect than the other two buttons. The active button is on the far right as the user is expected to press it the most. The user cannot use the edit and delete actions on an active goal. The system and UI will update instantaneously to reflect these changes. 
In figure \ref{fig:goalviewnofab} the user has scrolled the list so the FAB disappears for an unobstructed view. The home screen, figure \ref{fig:maingoalfull} contains information about the active goal, with the progress being representing both with a pie chart and text. The pie chart was made using a third party library called MPAndroidChart.
The user can choose the other FAB button option to add progress, figure \ref{fig:addprogress}. The non active goals can be edited by pressing the pencil icon, figure \ref{fig:editgoal}.

\begin{figure}[!htb]
\minipage{0.32\textwidth}
  \includegraphics[width=\linewidth]{pics/addgoal.png}
  \caption{Add goal dialogue}
  \label{fig:addgoal}
\endminipage\hfill
\minipage{0.32\textwidth}
  \includegraphics[width=\linewidth]{pics/goalviewfab.png}
  \caption{RecyclerView with FAB}\label{fig:goalviewfab}
\endminipage\hfill
\minipage{0.32\textwidth}%
  \includegraphics[width=\linewidth]{pics/goalviewnoFAB.png}
  \caption{RecyclerView after scroll}\label{fig:goalviewnofab}
\endminipage
\end{figure}
\begin{figure}[!htb]
\minipage{0.32\textwidth}
  \includegraphics[width=\linewidth]{pics/addprogress.png}
  \caption{Add progress dialogue}
  \label{fig:addprogress}
\endminipage\hfill
\minipage{0.32\textwidth}
  \includegraphics[width=\linewidth]{pics/editgoal.png}
  \caption{Edit goal dialogue}\label{fig:editgoal}
\endminipage\hfill
\minipage{0.32\textwidth}%
  \includegraphics[width=\linewidth]{pics/maingoalfull.png}
  \caption{Home screen}\label{fig:maingoalfull}
\endminipage
\end{figure}

\subsection{Basic activity history}

The user can view the history in the history fragment. They can see the progress of each day versus the goal value with the goal's name and date. The graph is made using a third party library called MPAndroidChart. The user can clear the history via the screen in figure \ref{fig:delhis}. The button opens up a alert dialogue to confirm the user's intention to clear the history.

\begin{figure}[!htb]
\minipage{0.32\textwidth}
  \includegraphics[width=\linewidth]{pics/graph.png}
  \caption{History screen}
  \label{fig:graph}
\endminipage\hfill
\minipage{0.32\textwidth}%
  \includegraphics[width=\linewidth]{pics/delhis.png}
  \caption{Delete history screen}\label{fig:delhis}
\endminipage
\end{figure}

\subsection{Test mode}

The user can enable test mode in the settings. With test mode on they can now choose, in addition to the normal parameters, the number of ``steps'' taken and the date of the goal as seen in figure \ref{fig:testmode}. When in test mode there is a small warning that appears next to the toolbar's title to remind the user they are in test mode.


\subsection{Enhanced activity goal setting and recording}

The user can choose the units to be associated with a goal and can add progress in any units they want, figures \ref{fig:stepsunits} and \ref{fig:goalsunits}. The progress is all stored as ``steps'' in the backend and converted when needed.

\begin{figure}[!htb]
\minipage{0.32\textwidth}
  \includegraphics[width=\linewidth]{pics/testmode.png}
  \caption{Test mode add goal dialogue}
  \label{fig:testmode}
\endminipage\hfill
\minipage{0.32\textwidth}
  \includegraphics[width=\linewidth]{pics/stepsunits.png}
  \caption{Units dropdown, Steps}
  \label{fig:stepsunits}
\endminipage\hfill
\minipage{0.32\textwidth}
  \includegraphics[width=\linewidth]{pics/goalsunits.png}
  \caption{Units dropdown, Goal}\label{fig:goalsunits}
\endminipage
\end{figure}

\subsection{Enhanced activity history}

The user can now specify date ranges, figure \ref{fig:dategraph}, percentage ranges, whether the goal was fully completed goals or not, figure \ref{fig:percentgraph}, and choose units, figure \ref{fig:graphunits}, when viewing a portion of the history. 

\begin{figure}[!htb]
\minipage{0.32\textwidth}
  \includegraphics[width=\linewidth]{pics/dategraph.png}
  \caption{DatePicker dialogue}
  \label{fig:dategraph}
\endminipage\hfill
\minipage{0.32\textwidth}
  \includegraphics[width=\linewidth]{pics/percentgraph.png}
  \caption{Percentage selection}\label{fig:percentgraph}
\endminipage\hfill
\minipage{0.32\textwidth}
  \includegraphics[width=\linewidth]{pics/graphunits.png}
  \caption{Units dropdown, History}\label{fig:graphunits}
\endminipage
\end{figure}

\subsection{Activity and goal statistics}

An alternate view of the statistics is available. It can be constrained by dates, by percentage ranges, figure \ref{fig:statscontraint} and data can be displayed in any units, figure \ref{fig:statsunits}. Statistics uses a date picker like in history to find the dates. Generated from the results are the values Average, Total, Maximum and Minimum. The goals are available to view at the bottom of the screen in a grid layout. 

\begin{figure}[!htb]
\minipage{0.32\textwidth}
  \includegraphics[width=\linewidth]{pics/statsnormal.png}
  \caption{Statistics screen}
  \label{fig:statsnormal}
\endminipage\hfill
\minipage{0.32\textwidth}
  \includegraphics[width=\linewidth]{pics/statscontraint.png}
  \caption{Percentage goal constrained}\label{fig:statscontraint}
\endminipage\hfill
\minipage{0.32\textwidth}%
  \includegraphics[width=\linewidth]{pics/statsunits.png}
  \caption{Units dropdown}\label{fig:statsunits}
\endminipage
\end{figure}

\subsection{User preferences and settings}

The user is given the freedom to customise the mapping between the steps and imperial and metric units and for Statistics and History, the default presentation view, which is the default length of time the history and statistics screens use when displaying historical data, the default units, and default percentage range information, figure \ref{fig:prefs2}. They can turn on or off test mode, step counter, notifications and the ability to edit goals using switches, figure\ref{fig:prefs1}. 

\begin{figure}[!htb]
\minipage{0.32\textwidth}
  \includegraphics[width=\linewidth]{pics/prefs1.png}
  \caption{Settings}
  \label{fig:prefs1}
\endminipage\hfill
\minipage{0.32\textwidth}
  \includegraphics[width=\linewidth]{pics/pref2.png}
  \caption{Settings}\label{fig:prefs2}
\endminipage
\end{figure}

\subsection{User notifications}

When the progress reaches the goal value the app will issue a notification to the user, figures \ref{fig:nots1} \& \ref{fig:notSmall}. 

\begin{figure}[!htb]
\minipage{0.32\textwidth}
  \includegraphics[width=\linewidth]{pics/nots.png}
  \caption{Notification drawer}
  \label{fig:nots1}
\endminipage\hfill
\minipage{0.32\textwidth}
  \includegraphics[width=\linewidth]{pics/notsmall.png}
  \caption{Notification icon}\label{fig:notSmall}
\endminipage\hfill
\minipage{0.32\textwidth}
  \includegraphics[width=4.3cm,height=3cm]{pics/fab.png}
  \caption{Expanded FAB}\label{fig:fabex}
\endminipage
\end{figure}

\subsection{Step detection}

The app uses the Step Detector sensor to detect when the user has taken a step. When a step is detected a counter is incremented and stored in the database. 

\subsection{Floating Action Button}

The FAB used in the app is the clans FAB. It is a third party library that easily allows the addition of multiple options, figure \ref{fig:fabex} to the FAB. 

\subsection{Warnings}

The app constrains the user so that they don't enter data that wouldn't work well in the system. The only place the user can input custom data is in the add goal and progress dialogues so this is where the error warnings are found. The user can only add numbers, no blanks, in the goal value, figures \ref{fig:addgoalerror} \& \ref{fig:addstepserror} and steps edit text views and each goal must have a unique name that isn't blank, \ref{fig:goalnameerror}. The errors are displayed using TextView.setError().


\begin{figure}[!htb]
\minipage{0.32\textwidth}
  \includegraphics[width=\linewidth]{pics/addgoalerror.png}
  \caption{Goal value error}
  \label{fig:addgoalerror}
\endminipage\hfill
\minipage{0.32\textwidth}
  \includegraphics[width=\linewidth]{pics/addstepserror.png}
  \caption{Steps value error}\label{fig:addstepserror}
\endminipage\hfill
\minipage{0.32\textwidth}
  \includegraphics[width=\linewidth]{pics/goalnameerror.png}
  \caption{Goal name error}\label{fig:goalnameerror}
\endminipage
\end{figure}

\subsection{Toasts and Snackbars}

Toasts are displayed to notify the user of the success or lack there of a particular action that was expected to change the state of the app, figure \ref{fig:toastnodelete}. Snackbars occur when the user has performed a permanent change and are given the change to reverse it. The snackbar has an UNDO button which has a listener attached, figure \ref{fig:snackbardel}. This contains the data that is to be changed, a deleted goal in this case, and when activated adds the goal back into the database.

\begin{figure}[!htb]
\minipage{0.32\textwidth}
  \includegraphics[width=\linewidth]{pics/toastnodelete.png}
  \caption{Toast message}
  \label{fig:toastnodelete}
\endminipage\hfill
\minipage{0.32\textwidth}
  \includegraphics[width=\linewidth]{pics/optionsmenu.png}
  \caption{Options Menu}\label{fig:menu}
\endminipage\hfill
\minipage{0.32\textwidth}
  \includegraphics[width=\linewidth]{pics/snackbardel.png}
  \caption{Snackbar message}\label{fig:snackbardel}
\endminipage
\end{figure}

\subsection{Options menu}

The Settings, Statistics and Clear History screens can be accessed through the options menu, figure \ref{fig:menu}

\section{Feedback Consideration}

The take away from the feedback of stage 1 was that more thought needed to be point in about the location of statistics and settings. They were put in the options as they were considered to be places the user wouldn't be navigating to as often as the three main fragments. 
From stage 2, it was suggested to put in a warning in the toolbar so the user was reminded they were in testmode and to use a card layout with a recycler view in the goals view as this is much cleaner when displaying a lot of information compared to a list view. The colour scheme was much improved, errors and feedback were implemented so that user could understand what was happening. Empty states were implemented for the main screens. The FAB was made to disappear on list scroll. 

\section{Third party libraries}

Two libraries were used.
\newline
Clan's Floating Action button: 
\url{https://github.com/Clans/FloatingActionButton}
\newline
PhilJay's MPAndroidChart: 
\url{https://github.com/PhilJay/MPAndroidChart}
\newline
\newline
To use, please ensure the following lines are in the app's gradle file:
\newline
    compile 'com.github.clans:fab:1.6.4'
    \newline
    compile 'com.github.PhilJay:MPAndroidChart:v3.0.1'
\end{document}